\chapter{How do i write unit tests?}
\hypertarget{md__2_users_2spirosmag_2_documents_2_algo_plus_2_u_n_i_t___t_e_s_t_s}{}\label{md__2_users_2spirosmag_2_documents_2_algo_plus_2_u_n_i_t___t_e_s_t_s}\index{How do i write unit tests?@{How do i write unit tests?}}
\label{md__2_users_2spirosmag_2_documents_2_algo_plus_2_u_n_i_t___t_e_s_t_s_autotoc_md89}%
\Hypertarget{md__2_users_2spirosmag_2_documents_2_algo_plus_2_u_n_i_t___t_e_s_t_s_autotoc_md89}%
 Writting unit tests is the most important part of contributing. Fortunately or not, this is a project about algorithms and a lot of edge cases exist, so we must make sure everything works as it should.


\begin{DoxyCode}{0}
\DoxyCodeLine{\textcolor{preprocessor}{\#define\ CATCH\_CONFIG\_MAIN}}
\DoxyCodeLine{\textcolor{preprocessor}{\#include\ "{}../catch2/catch.hpp"{}}}
\DoxyCodeLine{}
\DoxyCodeLine{TEST\_CASE(\textcolor{stringliteral}{"{}testing\ this\ and\ that..."{}})\{}
\DoxyCodeLine{\ \ \ \ std::vector<int>\ v\ =\ myclass.function();}
\DoxyCodeLine{\ \ \ \ some\ mode\ code...}
\DoxyCodeLine{\ \ \ \ std::vector<int>\ check\_v\ =\ \{the\ answer\ that\ myclass.function\ should\ \textcolor{keywordflow}{return}\};}
\DoxyCodeLine{\ \ \ \ REQUIRE(check\_v\ ==\ v);}
\DoxyCodeLine{\}}

\end{DoxyCode}
 